\usepackage{xparse}
\DeclareDocumentCommand{\entry}{ O{} O{} m m m m} {
  \newglossaryentry{gls-#3}{name={#5},text={#5\glsadd{#3}},
    description={#6},#1
  }
  \makeglossaries
  \newacronym[see={[Siehe:]{gls-#3}},#2]{#3}{#4}{#5\glsadd{gls-#3}}
}


\newglossaryentry{defacement}
{
  name=Defacement,
  description={(engl. für „Entstellung“ oder „Verunstaltung“) bezeichnet das unberechtigte Verändern einer Website}
}

\newglossaryentry{frontend}
{
	name=Front End,
    description={die der Öffentlichkeit zugänglichen Internetseiten}
}

\newglossaryentry{backend}
{
	name=Back End,
    description={die Administrationsoberfläche zum Erstellen und Pflegen von Inhalten einer Webseite}
}

\newglossaryentry{unix}
{
	name=UNIX,
    description={universell einsetzbares, besonders leistungsfähiges Betriebssystem}
}

\newglossaryentry{linux}
{
	name=Linux,
    description={freies Betriebssystem, das \gls{unix} basierend ist}
}

\newglossaryentry{widget}
{
	name=Widget,
    description={kleines Computerprogramm, das in ein anderes Programm integriert wird, besonders als Teil einer grafischen Benutzeroberfläche}
}

\newglossaryentry{zip}
{
	name=ZIP,
    description={ein Format für verlustfrei komprimierte Dateien}
}

\newglossaryentry{opensource}
{
	name={Open Source},
    description={Software bezeichnet, deren Quelltext öffentlich und von Dritten eingesehen werden kann}
}

\entry{pixel}{px}{Pixel}{von Picture Element, Bildelement}{}
\entry{tinymce}{TinyMCE}{Tiny Moxiecode Content Editor}{Editor für Webanwendungen}
\entry{ip}{IP}{Internet Protocol}{in Computernetzen weit verbreitetes Netzwerkprotokoll, stellt die Grundlage des Internets dar}
\entry{php}{PHP}{PHP: Hypertext Preprocessor}{eine Skriptsprache, die hauptsächlich zur Erstellung dynamischer Webseiten oder Webanwendungen verwendet wird}
\entry{html}{HTML}{Hypertext Markup Language}{eine textbasierte Auszeichnungssprache zur Strukturierung digitaler Dokumente wie Texte mit Hyperlinks, Bildern und anderen Inhalten}
\entry{css}{CSS}{Cascade Style Sheet}{eine Computersprache für die Gestaltung digitaler, vorwiegend Web-basierter Dokumente}
\entry{cms}{CMS}{Content Management System}{eine Software zur gemeinschaftlichen Bearbeitung von ''Inhalten'' (Content), zumeist in Webseiten}
\entry{sql}{SQL}{Structured Query Language}{eine Datenbanksprache zur Definition von Datenstrukturen in relationalen Datenbanken sowie zum Bearbeiten (Einfügen, Verändern, Löschen) und Abfragen von darauf basierenden Datenbeständen}
\entry{http}{HTTP}{Hypertext Transfer Protocol}{ein Anwendungsprotokoll zur Übertragung von Daten über ein Netzwerk}
\entry{iis}{IIS}{Internet Information Services}{ein erweiterbarer Webserver für Windows}
\entry{rss}{RSS}{Rich Site Summary}{eine Technologie zum Abonnement von Webseiten-Inhalten}
\entry{www}{WWW}{World Wide Web}{ein über das Internet abrufbares System von elektronischen Hypertext-Dokumenten}
\entry{ldap}{LDAP}{Lightweight Directory Access Protocol}{ein Netzwerkprotokoll zur Abfrage und Änderung von Informationen verteilter Verzeichnisdienste}
\entry{gif}{GIF}{Grafics Interchange Format}{ein Grafikformat für Bilder mit Farbpalette}
\entry{png}{PNG}{Portable Network Graphics}{en Grafikformat, das als Ersatz für das Format GIF entworfen wurde}
\entry{wysiwyg}{WYSIWIG}{What You See Is What You Get}{bei WYSIWYG wird ein Dokument während der Bearbeitung am Bildschirm genauso angezeigt, wie es bei der Ausgabe über ein anderes Gerät, z. B. einen Drucker, aussieht}
\entry{ftp}{FTP}{File Transfer Protocol}{ein Netzwerkprotokoll zur Dateiübertragung}
\entry{pdf}{PDF}{Portable Document Format}{ein plattformunabhängiges Dateiformat für Dokumente}
\entry{rwd}{RWD}{responsives Webdesign}{ein gestalterisches und technisches Muster zur Erstellung von Websites, so dass diese auf Eigenschaften des jeweils benutzten Endgeräts, vor allem Smartphones und Tabletcomputer, reagieren können}
\entry{url}{URL}{Uniform Resource Locator}{identifiziert und lokalisiert eine Ressource, beispielsweise eine Website über die zu verwendende Zugriffsmethode (zum Beispiel das verwendete Netzwerkprotokoll wie \acrshort{http} oder \acrshort{ftp}) und den Ort der Ressource in Computernetzwerken}